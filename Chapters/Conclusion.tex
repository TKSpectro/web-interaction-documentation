\chapter{Fazit}
\label{Conclusion}

Das Ziel meiner Arbeit war es verschiedene Technologien und Libraries anhand eines praktisches Beispiels zu benutzen und im Bezug auf Accessabilityunterstütztung während der Entwicklung zu untersuchen. Hierbei sind im Besonderen die Libraries RadixUI und CmdK sehr positiv aufgefallen.

RadixUI nimmt dem Entwickler sehr viel manuelle Arbeit und Recherche ab, da es sinnvolle Defaults mit sich bringt und durch die WAI-ARIA konformität auch immer auf den aktuellsten Standards agiert. Für die Teile die RadixUI nicht automatisch generieren/übernehmen kann, gibt es in der Dokumentation sehr gute Beispiele und Hinweise, die es dem Entwickler ermöglichen, die Accessability Features selbst zu implementieren. Dies ist insbesondere für die Nutzung von ARIA-Attributen sehr hilfreich, da diese sehr umfangreich sind und somit auch sehr viel Vorwissen bzw. extra Recherchen erfordern.

CmdK bietet eine Combobox Menü Implementierung, welche auf RadixUIs Dialog basiert und somit Accessability Features wie z.B. Tastatursteuerung und Screenreader Unterstützung direkt mitbringt. Meiner Meinung nach sind diese Art von Menüs unglaublich sinnvoll und sollten in Zukunft, vorallem auch auf nicht Technologisch orientierten Webseiten und Tools, vermehrt eingesetzt werden. Sie bieten eine unkomplizierte und vorallem effiziente Möglichkeit, Aktionen und Features zu suchen und zu nutzen. Hierbei ist auch der Nutzen für Screenreader Nutzer sehr groß, da diese nun nicht mehr durch eine Menüstruktur navigieren müssen, um an die gewünschte Aktion zu gelangen.

Analyse Tools wie Lighthouse sind während der Entwicklung sehr gut einsetzbar, da Sie per Knopfdruck direkt Auswertungen zu Accessability Punkten liefern. Außerdem geben sie bei Problemen direkt Hinweise, wie diese behoben werden können. Dies ist insbesondere für Entwickler, die wenig Erfahrung mit Accessability Features haben, sehr hilfreich.
Damit diese Tests automatisiert werden können und somit auch in einem CI/CD Prozess integriert werden können, gibt es Tools wie Axe oder die Lighthouse CLI, welche die Auswertungen in einem JSON Format liefern. Diese können dann in einem CI/CD Prozess verwendet werden, um die Accessability Features zu testen und zu überprüfen.

Auch wenn Bibliotheken wie RadixUI Entwicklern viele a11y Features direkt mitbringen, ist es dennoch wichtig, sich mit Accessability Features und deren Implementierung auseinanderzusetzen. Dafür kann ich den, im Kapitel \ref{sec:research_apg} angesprochenen, \emph{ARIA Authoring Practices Guid} sehr empfehlen. Denn dieser gibt einen sehr guten Überblick über die verschiedenen ARIA-Features und deren Implementierung. Außerdem gibt er sehr gute Beispiele und Hinweise, wie diese Features in der Praxis angewendet werden können. Auch die Kapitel zu Bennung und Beschreibung von Elementen sind sehr hilfreich, da dies ein sehr wichtiger Aspekt der Accessability ist.