\chapter{Einführung} 
\label{Introduction}

In den letzten Jahren hat die Bedeutung von Web-Accessibility, also der Zugänglichkeit von Webinhalten für Menschen mit Behinderungen, stark zugenommen. Das Internet bietet eine Vielzahl von Informationen und Dienstleistungen, die für Menschen mit eingeschränkten körperlichen oder geistigen Fähigkeiten von großer Bedeutung sind. Es ist daher von entscheidender Bedeutung, dass Web-Entwickler ihre Websites barrierefrei gestalten, um sicherzustellen, dass alle Benutzer sie problemlos nutzen können.

Zum Glück gibt es heute eine Vielzahl von Tools, die Entwicklern dabei helfen können, barrierefreie Websites zu erstellen. Diese reichen von einfachen Validierungs-Tools, die überprüfen, ob eine Website bestimmte Standards erfüllt, bis hin zu komplexen Werkzeugen, die sich speziell auf die Bedürfnisse von Menschen mit Behinderungen konzentrieren.

In dieser Ausarbeitung wird sich mit einigen der wichtigsten Tools und Techniken beschäftigen, die Entwicklern zur Verfügung stehen, um sicherzustellen, dass ihre Websites barrierefrei sind. Es wird sich unter anderem auch mit der Beispielhaften Umsetzung einer barrierefreien Website beschäftigt und anhand dieser verschiedene Techniken und Libraries vorgestellt.